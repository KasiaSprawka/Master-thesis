\pagenumbering{arabic}
	\setcounter{page}{1}
	
\chapter*{Introduction}
	\addcontentsline{toc}{chapter}{Introduction}

\begin{comment}	
\lettrine[lines=3, slope=1em, findent=-0.8em]{A}{ccording to the beliefs of ancient Hebrews}, the kidneys were \cite{maio1999metaphorical}. Today, more mundane, but not less important functionality is being assigned to them. 

Kidneys, although often underestimated, are fundamental organs of human body and their working mechanism is extremely complex. Their essential task is to remove wastes from the organism but their functionality is much wider. They are also involved in maintaining acid-based balance, regulating the blood pressure and are major endocrine organs, which secret three important hormones: erythropoietin, calcitriol and renin. Besides the production, they also take part in degradation of hormones such as insulin or parathyroid hormon \cite{saladin}.

Gradually progressing loss of kidney function known as a chronic kidney desease is a~ growing world-wide problem. As much as 8--16\% of whole population suffers from this condition \cite{statistics}. It significantly decreases comfort of life and in extreme cases leads to death. What is more, it was shown that renal diseases are risk factor for development of cardiovascular diseases \cite{cardiovascular_diseases}.
Because of the fact that symptoms don't resemble renal failure, approximately 90\% of the ill are unconscious of it until late stages \cite{national_kidney_foundation}. That is way there is the demand for methods, which enable fast and accurate measurement of renal function required for all of three: prevention, monitoring and therapy.
The metrics of level of kidney function is glomelural filtration rate (GFR) \cite{traynor2006measure}. Not only does it allow for assessment how well our kidneys are working, but also it can determine the stage of kidney disease.
\end{comment}