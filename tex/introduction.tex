
\pagenumbering{arabic}
	\setcounter{page}{1}
	
%\fancyhead[LE,RO]{{\small{\author}}\linebreak{\small\textsl{\title}}}
%\fancyhead[RE,LO]{}
%\renewcommand{\headrulewidth}{1pt}
	
\chapter{Introduction}
	%\addcontentsline{toc}{chapter}{Introduction}

%To be done


\lettrine[lines=3, slope=1em, findent=-0.8em]{A}{ccording to the beliefs of ancient Hebrews}, the kidneys are the seat of the human soul and consciousness. They were also assosiated with the felling of the fear and sadness \cite {maio1999metaphorical}. Today, more mundane, but not less important tasks are being assigned to them. 

Kidneys, although often underestimated, are the fundamental organs of human body and their working mechanism is extremely complex. Their essential task is to remove wastes from the organism but their functionality is much wider. They are also involved in maintaining acid-based balance, regulating the blood pressure and are major endocrine organs, which secret three important hormones: \textit{erythropoietin}, \textit{calcitriol} and \textit{renin} \cite{saladin}. In short terms, they maintain whole body homoeostasis, which is essential for overall health of the organism. 

Gradually progressing loss of kidney function known as a chronic kidney desease is a~ growing world-wide problem. As much as 8--16\% of whole population suffers from this condition \cite{statistics}. It significantly decreases comfort of life and in extreme cases leads to death. What is more, it was shown that renal diseases are risk factor for development of cardiovascular diseases \cite{cardiovascular_diseases}.
Because of the fact that symptoms do not resemble renal failure, approximately 90\% of the ill are unconscious of it until late stages \cite{national_kidney_foundation}. That is why there is the demand for methods, which enable fast and accurate measurement of renal function required for all of three: prevention, monitoring and therapy.

The metrics of the level of kidney function is \textit{glomelural filtration rate} (GFR) \cite{traynor2006measure}. Good performance of the several important functions of the kidney are dependent on the GFR value. Not only does it allow for assessment how well our kidneys are working, but also it can determine the stage of kidney disease.
The gold standard of GFR measurement incorporates injection of the exogenous marker that is freely filtered by the kidney, and that does not undergo metabolism, tubular secretion or absorption. An example of such a~marker can be insulin.
However, in clinical practice usually the endogenous marker is used such a creatinine or urea and GFR is estimated applying robust algoritms \cite{delanaye2012measuring}.
Although chemical methods allow for accurate GFR estimation, they are not very practical in clinical use. Not only are they time-consuming and expensive but also they can be cumbersome. What is more they provide information about combined GFR value and cannot be used for a single kidney function assessment. Thus, other methods are desired \cite{bokacheva2008assessment}.

An innovative approach in estimating renal function is performing dynamic contrast-enhanced magnetic resonance (DCE-MRI), which provides time-varying images of the abdominal.
The analysis of the obtained time-intensity changes as a function of time provides important information about renal performance \cite{bokacheva2008assessment, khalifa2014models}. Traditionally, this evaluation is performed by experienced observer, although this method is very subjective and strongly depends on the experience of the expert. Other technique involves fitting tissue intensity changes to \textit{pharmacokinetic} (PK) models, which allows quantification of renal function \cite{khalifa2014models}. Even though this strategy is gaining more and more supporters, most of the methods still require interference of the human at some stage, which makes them vulnerable to human factors. 

\section{Aims and scope of the thesis}
The works included in this thesis are the part of the bigger project realised at the University of Bergen in Norway, which aims to develop entirely data-driven method of GFR estimation directly from DCE-MRI, which would be fast and efficient and accurate enough to be used it clinical applications.  
 
The overarching aims of this thesis is to create the library for GFR estimation by pharmacokinetic modelling in Python programming language and to compare the performance of different PK models in renal function estimation. What is more, it should be decided if any of them is accurate enough to be used in the target method of fast GFR estimation. 

The structure of this thesis is as follows:
because of the fact that the thesis is connected with the assessment of the renal function Chapter 2 deals with the basics of the renal anatomy and physiology in order to understand the mechanism of glomerural filtration and importance of the metric, which is GFR.
Chapter 3 focuses on the basic principles of operation of DCE-MRI and introduces the current trends and obstacles in its examination.  
Furthermore, Chapter 4 is devoted to the issue of quantitative analysis of DCE-MRI. 
It discusses the  basics of tracer kinetic theory and lists all components required for pharmacokinetic modelling.  
Next, Chapter 5 and Chapter 6 focus on the practical part of the work.  
Chapter 5 describes in details the subsequent steps of DCE-MRI processing, whereas Chapter 6 presents obtained results. 
Finally, in Chapter 7, the obtained results are discussed and the comparison of different PK models is covered. 


