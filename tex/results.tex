\section{Results}
The developed algorithm was evaluated on ten MRI sequences of ten different participants. For every of the participant one of the two  MRI sequences was chosen randomly.

The segmentation of the pelvis was evaluated with \textit{Dice Similarity Coefficient} (DSC) given by the formula:
\begin{equation}
	\label{eq:dice}
	DSC = \dfrac{2|A\cap{}B|}{|A|+|B|},
\end{equation}
where |A| and |B| are the number of voxels in pelvis detected  in automatic segmentation and the number of voxels in ground truth labelled organoleptically, respectively.
The average accuracy of the pelvis segmentation was equal to DSC~=~0.87$\pm$0.06.



\begin{landscape}
\begin{table}
\centering
\caption[Clinical characteristic of the participants]{Clinical characteristic of the participant \cite{eikefjord2017dynamic}.}
\label{tab:results}
\begin{threeparttable}
\rowcolors{3}{}{middleblue!30}
\renewcommand{\arraystretch}{1.5}
\begin{tabular}{l r r c r r r c r r r c r r r c r r r}
	\toprule
	%\multirow{3}{*}{\shortstack[l]{\textbf{Subject}\\ \textbf{no.}}}&
	\multirow{3}{*}{\textbf{No.}}&
	\multirow{3}{*}{\textbf{mGFR}}&
	\multirow{3}{*}{\textbf{eGFR}}& \phantom{abc}&
	\multicolumn{15}{c}{\textbf{MRI GFR}}\\ \cmidrule{5-19}
	&&&& \multicolumn{3}{c}{\textbf{TK}} & \phantom{abc}&
    \multicolumn{3}{c}{\textbf{ETK}} & \phantom{abc}&
    \multicolumn{3}{c}{\textbf{PR}} & \phantom{abc}&
    \multicolumn{3}{c}{\textbf{2CXM}}\\
    \cmidrule{5-7} \cmidrule{9-11} \cmidrule{13-15} \cmidrule{17-19}
   \rowcolor{white} &&&& \textbf{L} & \textbf{R} & \textbf{T} & & \textbf{L} & \textbf{R} & \textbf{T} && \textbf{L} & \textbf{R} & \textbf{T} && \textbf{L} & \textbf{R} & \textbf{T}\\ 
    \toprule
    %\rowcolors{2}{}{beaublue!50}
  	1  & 107 & 138 & & 75 & 67 & 142 & & 63 & 55 & 118 & & &&&&&&\\
  	2  & 98  & 109 & & 75 & 32 & 137 & & 61 & 53 & 115 & & &&&&&&\\
  	3  & 90  & 108 & & 48 & 57 & 106 & & 42 & 50 & 92  & & &&&&&&\\
  	4  & 93  & 107 & & 55 & 49 & 105 & & 47 & 42 & 89  & & &&&&&&\\
  	5  & 94  & 83  & & 71 & 74 & 145 & & 61 & 62 & 123 & & &&&&&&\\
  	6  & 103 & 107 & & 42 & 48 & 90  & & 34 & 40 & 74  & & &&&&&&\\
  	7  & 112 & 125 & & 48 & 52 & 100 & & 36 & 39 & 75  & & &&&&&&\\
  	8  & 119 & 142 & & 50 & 61 & 111 & & 38 & 47 & 85  & & &&&&&&\\
  	9  & 96  & 110 & & 44 & 58 & 102 & & 35 & 52 & 87  & & &&&&&&\\
  	10 & 112 & 133 & & 59 & 64 & 123 & & 46 & 50 & 96  & & &&&&&&\\

  \bottomrule

\end{tabular}
\begin{tablenotes}%
\footnotesize{}%
\item L and R are SKGFR for left and right kidneys respectively whereas T is total GFR.
\item Note that $L+R = T$ is not always satisfied because of the rounding error.
\item All GFR values are expressed in ml/min$^{-1}$/1.73\,m$^2$.
    \end{tablenotes}
	\end{threeparttable}
\end{table}
\end{landscape}

\begin{table}
\centering
\caption{results2}
\label{tab:results2}
\begin{threeparttable}
\rowcolors{2}{}{middleblue!30}
\renewcommand{\arraystretch}{1.5}
\begin{tabular}{L{3cm} R{3cm} R{3cm} R{3cm}}
	\toprule

	{\textbf{Method}} & \textbf{Absolute error}  & \textbf{R$^2$} & \textbf{P30} \\ \toprule
				 TK   & 		      			 &				  &              \\
				ETK   & 		      			 &				  & \\
				 PK   & 		      			 &				  &\\
			    2CXM  & 		      			 &				  &\\
				eGFR  & 		      			 & \cellcolor{gray}				  &\\

  	\bottomrule

\end{tabular}
\begin{tablenotes}%
\footnotesize{}%
\item L and R are SKGFR for left and right kidneys respectively whereas T is total GFR.
\item Note that $L+R = T$ is not always satisfied because of the rounding error.
\item All GFR values are expressed in ml/min$^{-1}$/1.73\,m$^2$.
    \end{tablenotes}
	\end{threeparttable}
\end{table}
