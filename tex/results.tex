\section{Results}
The developed algorithm was evaluated on ten MRI sequences of ten different participants. For every of the participant one of the two  MRI sequences was chosen randomly.

The segmentation of the pelvis was evaluated with \textit{Dice Similarity Coefficient} (DSC) given by the formula:
\begin{equation}
	\label{eq:dice}
	DSC = \dfrac{2|A\cap{}B|}{|A|+|B|},
\end{equation}
where |A| and |B| are the number of voxels in pelvis detected  in automatic segmentation and the number of voxels in ground truth labelled organoleptically, respectively.
The average accuracy of the pelvis segmentation was equal to DSC~=~0.87$\pm$0.06.


\begin{landscape}
\begin{table}
\centering
\caption[Clinical characteristic of the participants]{Clinical characteristic of the participant \cite{eikefjord2017dynamic}.}
\label{tab:results}
\begin{threeparttable}
\rowcolors{4}{}{beaublue!50}
\renewcommand{\arraystretch}{1.25}
\begin{tabular}{c c c c c c c c c c c c c c c}
	\toprule
	\multirow{2}{*}{\textbf{Subject no.}}&
	\multirow{2}{*}{\textbf{mGFR}}&
	\multirow{2}{*}{\textbf{eGFR}}&
	\multicolumn{3}{c}{\textbf{TK}} &
    \multicolumn{3}{c}{\textbf{ETK}} &
    \multicolumn{3}{c}{\textbf{PR}}&
    \multicolumn{3}{c}{\textbf{2CXM}}\\
    &&& \textbf{L} & \textbf{P} & \textbf{T}  & \textbf{L} & \textbf{P} & \textbf{T} & \textbf{L} & \textbf{P} & \textbf{T} & \textbf{L} & \textbf{P} & \textbf{T}\\ 
    \hline
    
    %\rowcolors{2}{}{beaublue!50}
  	1 &&&&&&&&&&&&&&\\
  	2 &&&&&&&&&&&&&&\\
  	3 &&&&&&&&&&&&&&\\
  	4 &&&&&&&&&&&&&&\\
  	5 &&&&&&&&&&&&&&\\
  	6 &&&&&&&&&&&&&&\\
  	7 &&&&&&&&&&&&&&\\
  	8 &&&&&&&&&&&&&&\\
  	9 &&&&&&&&&&&&&&\\
  	10 &&&&&&&&&&&&&&\\

  \bottomrule

\end{tabular}
\begin{tablenotes}%
\footnotesize{}%
\item Values in parentheses are ranges.
\item Plus minus values are means $\pm$ Standard Deviations (SD).
\item L R are SKGFR for left and right kidneys respectively and T is total GFR
    \end{tablenotes}
	\end{threeparttable}
\end{table}
\end{landscape}