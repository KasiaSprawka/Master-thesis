% 
% Plik zawierający załączniki
%
	
	% Otoczenie dla załączników. 
	% #1 to etykieta do odwołań \ref
	% #2 określa, czy załącznik ma być widoczny (\visibletrue lub \visiblefalse)
	\newcounter{zalacznik}
	\renewcommand{\thezalacznik}{\Alph{zalacznik}}
	\NewEnviron{zalacznik}[2] {%
		\newif \ifvisible
		#2%
		\ifvisible
			\newpage%
			\refstepcounter{zalacznik}\label{#1}%
			\section*{Załącznik~\thezalacznik}%
			\vfill\BODY\vfill
		\fi
	}
	
%░░░░░░░░░░░░░░░░░░░░░░░░░░░░░░░░░░░░░░░░░░░░░░░░░░▒▒▒▒▒▒▒▒▒▒▒▒▒▒▒▒▒▒▒▒░░░░░░░░░░░░░░░░░░░░
	\begin{zalacznik}{appx:TabelaParametrowKlatt}{\visibletrue}
	\def\objasnienieCV{Parametry oznaczone symbolem ,,V" mogą być zmieniane dynamicznie.\\Parametry oznaczone symbolem ,,C" mają stałą wartość i~służą do ogólnej konfiguracji syntezatora.}	
	
		\begin{table}[!hbp]
			\centering
			\caption[Lista parametrów syntezatora Klatta]{Lista parametrów syntezatora Klatta~\cite{book:Styger}.}
			\label{tab:TabelaParametrowKlatt}
			\vspace*{6pt}
			\begin{minipage}{\textwidth}
				\centering
				\begin{tabular}{lllll}
					\toprule[1.2pt]
					Symbol & C\slash V\footnote{\objasnienieCV{}} & Min. & Max. & Nazwa \\
					\midrule
					DU & C & 30 & 5000 & Duration of the utterance (ms) \\
					NWS & C & 1 & 20 & Update interval for parameter reset (ms) \\
					SR & C & 5000 & 20000 & Output sampling rate (Hz)  \\
					NF & C & 1 & 6 & Number of formants in cascade branch \\
					SW & C & 0 & 1 & 0=Cascade, 1=Parallel tract excitation by AV \\
					G0 & C & 0 & 80 & Overall gain scale factor (dB) \\
					F0 & V & 0 & 500 & Fundamental frequency (Hz)  \\
					AV & V & 0 & 80 & Amplitude of voicing (dB)  \\
					AVS & V & 0 & 80 & Amplitude of quasi-sinusoidal voicing (dB) \\
					FGP & V & 0 & 600 & Frequency of glottal resonator RGP \\
					BGP & V & 50 & 2000 & Bandwidth of glottal resonator RGP \\
					FGZ & V & 0 & 5000 & Frequency of glottal anti-resonator RGZ \\
					BGZ & V & 100 & 9000 & Bandwidth of glottal anti-resonator RGZ \\
					BGS & V & 100 & 1000 & Bandwidth of glottal resonator RGS \\
					AH & V & 0 & 80 & Amplitude of aspiration (dB)  \\
					AF & V & 0 & 80 & Amplitude of frication (dB)  \\
					F1 & V & 180 & 1300 & Frequency of 1st formant (Hz) \\
					B1 & V & 30 & 1000 & Bandwidth of 1st formant (Hz) \\
					\ldots &&&& \ldots \\
	%				F2 & V & 550 & 3000 & Frequency of 2nd formant (Hz) \\
	%				B2 & V & 40 & 1000 & Bandwidth of 2nd formant (Hz) \\
	%				F3 & V & 1200 & 4800 & Frequency of 3rd formant (Hz) \\
	%				B3 & V & 60 & 1000 & Bandwidth of 3rd formant (Hz) \\
	%				F4 & V & 2400 & 4990 & Frequency of 4th formant (Hz) \\
	%				B4 & V & 100 & 1000 & Bandwidth of 4th formant (Hz) \\
	%				F5 & V & 3000 & 6000 & Frequency of 5th formant (Hz) \\
	%				B5 & V & 100 & 1500 & Bandwidth of 5th formant (Hz) \\
					F6 & V & 4000 & 6500 & Frequency of 6th formant (Hz) \\
					B6 & V & 100 & 4000 & Bandwidth of 6th formant (Hz) \\
					FNP & V & 180 & 700 & Frequency of nasal pole (Hz) \\
					BNP & V & 40 & 1000 & Bandwidth of nasal pole (Hz) \\
					FNZ & V & 180 & 800 & Frequency of nasal zero (Hz) \\
					BNZ & V & 40 & 1000 & Bandwidth of nasal zero (Hz) \\
					AN & V & 0 & 80 & Amplitude of nasal formant (dB) \\
					A1 & V & 0 & 80 & Amplitude of 1st formant (dB) \\
					A2 & V & 0 & 80 & Amplitude of 2nd formant (dB) \\
					\ldots &&&& \ldots \\
	%				A3 & V & 0 & 80 & Amplitude of 3rd formant (dB) \\
	%				A4 & V & 0 & 80 & Amplitude of 4th formant (dB) \\
	%				A5 & V & 0 & 80 & Amplitude of 5th formant (dB) \\
					A6 & V & 0 & 80 & Amplitude of 6th formant (dB) \\
					AB & V & 0 & 80 & Amplitude of bypass path (dB) \\
					\bottomrule[1.5pt]
				\end{tabular}\par
				\vspace{-0.75\skip\footins}
				\renewcommand{\footnoterule}{}
			\end{minipage}
		\end{table}
	\end{zalacznik}
	
%░░░░░░░░░░░░░░░░░░░░░░░░░░░░░░░░░░░░░░░░░░░░░░░░░░▒▒▒▒▒▒▒▒▒▒▒▒▒▒▒▒▒▒▒▒░░░░░░░░░░░░░░░░░░░░
	\begin{zalacznik}{appx:PlytkaSTM32F4}{\visibletrue}
		\begin{figure}[ht!]
			\centering
			\includegraphics[trim = 0.0cm 0.0cm 0.0cm 0.0cm, clip = false]{Ilustracje/ZdjecieSTM32F4.pdf}
			\caption{Płytka ewaluacyjna \textit{STM32F4Discovery}}
			\label{fig:PlytkaSTM32F4}
		\end{figure}
	\end{zalacznik}
	

%░░░░░░░░░░░░░░░░░░░░░░░░░░░░░░░░░░░░░░░░░░░░░░░░░░▒▒▒▒▒▒▒▒▒▒▒▒▒▒▒▒▒▒▒▒░░░░░░░░░░░░░░░░░░░░
	\begin{zalacznik}{appx:SchematFunkcjonalnySTM32F407VGT6}{\visiblefalse}
	
		\begin{figure}[ht!]
			\centering
			\includegraphics[trim = 0.0cm 0.0cm 0.0cm 0.0cm, clip = false, height = 18cm]{Ilustracje/SchematFunkcjonalnySTM32F407VGT6.pdf}
	%		\addtocounter{appendixfig}{1}
			\caption[Schemat funkcjonalny mikrokontrolera STM32F407VGT6]{Schemat funkcjonalny mikrokontrolera STM32F407VGT6}
	%		\addtocounter{figure}{-1}
			\label{fig:SchematFunkcjonalnySTM32F407VGT6}
		\end{figure}
		
	\end{zalacznik}
	
%░░░░░░░░░░░░░░░░░░░░░░░░░░░░░░░░░░░░░░░░░░░░░░░░░░▒▒▒▒▒▒▒▒▒▒▒▒▒▒▒▒▒▒▒▒░░░░░░░░░░░░░░░░░░░░
	\begin{zalacznik}{appx:SchematPolaczeniowyMCU}{\visibletrue}
	
		\begin{figure}[ht!]
			\centering
			\includegraphics[trim = 0.0cm 0.0cm 0.0cm 0.0cm, clip = false, width = 0.8\textwidth]{Ilustracje/SchematPolaczeniowyMCU.pdf}
	%		\addtocounter{appendixfig}{1}
			\caption[Schemat funkcjonalny mikrokontrolera STM32F407VGT6]{Schemat funkcjonalny mikrokontrolera STM32F407VGT6}
	%		\addtocounter{figure}{-1}
			\label{fig:SchematPolaczeniowyMCU}
		\end{figure}
		
	\end{zalacznik}
	
%░░░░░░░░░░░░░░░░░░░░░░░░░░░░░░░░░░░░░░░░░░░░░░░░░░▒▒▒▒▒▒▒▒▒▒▒▒▒▒▒▒▒▒▒▒░░░░░░░░░░░░░░░░░░░░
	\begin{zalacznik}{appx:eSpeakEditExtendedOkno}{\visibletrue}
		\begin{figure}[ht!]
			\centering
			\includegraphics[angle = 90, height = 0.88\textheight]{Ilustracje/eSpeakEditExtendedOkno43.png}
			\caption{Główne okno programu \textit{eSpeak--Edit Extended}}
			\label{fig:eSpeakEditExtendedOkno}
		\end{figure}
	\end{zalacznik}
	
%░░░░░░░░░░░░░░░░░░░░░░░░░░░░░░░░░░░░░░░░░░░░░░░░░░▒▒▒▒▒▒▒▒▒▒▒▒▒▒▒▒▒▒▒▒░░░░░░░░░░░░░░░░░░░░
	\begin{zalacznik}{appx:AlfabetIPA}{\visiblefalse}

		\begin{figure}[ht]
		\begin{minipage}[t]{15cm}
			\centering
			\includegraphics[height = 16cm]{Ilustracje/Pelny-alfabet-IPA.pdf}
	%		\caption []{Tabela alfabetu IPA.}
			\label{fig:Pelny-alfabet-IPA}
			\end{minipage}
		\end{figure}
	
	\end{zalacznik}
	
%░░░░░░░░░░░░░░░░░░░░░░░░░░░░░░░░░░░░░░░░░░░░░░░░░░▒▒▒▒▒▒▒▒▒▒▒▒▒▒▒▒▒▒▒▒░░░░░░░░░░░░░░░░░░░░
	\begin{zalacznik}{appx:ListaTagowSSML}{\visiblefalse}
	
		\begin{table}[h]
			\centering
			\caption{Lista rozpoznawanych tagów SSML.}
			\label{tab:ListaTagowSSML}
			\vspace*{6pt}
			\begin{tabular}{c p{5cm}}
				\toprule[1.2pt]
				Mnemonik & \\
				\midrule[1.2pt]
				$<$speak$>$ & xml:base \\ 
				$<$voice$>$ & • \\ 
				$<$prosody$>$ & • \\ 
				$<$say-as$>$ & • \\ 
				$<$mark$>$ name & • \\ 
				$<$s$>$ & • \\ 
				$<$p$>$ & • \\ 
				$<$sub$>$ alias & • \\ 
				$<$tts:style$>$ & • \\ 
				$<$audio$>$ src & • \\ 
				$<$emphasis$>$ & • \\ 
				$<$break$>$ & • \\
				\bottomrule[1.5pt]
			\end{tabular}
		\end{table}
		
	\end{zalacznik}

