%
% Streszczenie
%




	\newpage	
	\thispagestyle{empty}
	
	\phantomsection \label{sec:Streszczenie}
	\addcontentsline{toc}{chapter}{Streszczenie}
	
	\begin{spacing}{1.0}
	\ifantyplagiat
	\else
		\begin{center}
			%\vspace*{-3.35cm}
			{\scshape\normalsize Politechnika Łódzka} \\ 
			{\scshape\normalsize Wydział Elektrotechniki, Elektroniki, Informatyki i Automatyki} \\
			\vspace{0.5cm}	\textbf{\large\author{}} \\
			\vspace{0.3cm}
			{\scshape\footnotesize Praca dyplomowa magisterska} \\
			\vspace{0.25cm}
			\textbf{\large\titlepl{}} \\
			\vspace{0.25cm}
			{Łódź, 2018}
		\end{center}
		
		\begin{flushleft}
			{Opiekun: \supervisor{}}\\
			{Opiekun dodatkowy: \auxiliarySupervisor{}}\\
		\end{flushleft}
		\fi
		
		\begin{center}
			{\bf\large Streszczenie}
		\end{center}
	\end{spacing}
	
	
	\begin{spacing}{1.0}
	\begin{small}

Nerki są odpowiedzialne za utrzymanie homeostazy całego ciała umożliwiając organizmowi funkcjonowanie w optymalnych warunkach. Miarą wydolności nerek jest współczynnik przesączania kłębuszkowego (GFR), a jego monitorowanie jest niezbędne w~przewidywaniu, diagnostyce oraz leczeniu chorób nerek. Obecnie klinicznie używane metody chemiczne są kłopotliwe, a ponadto nie umożliwiają pomiaru GFR dla każdej nerki z osobna, w~przeciwieństwie do analizy obrazów rezonansu magnetycznego o kontraście dynamicznie wzmocnionym środkiem cieniującym (DCE-MRI), która umożliwia nieinwazyjną ocenę struktury oraz funkcjonowania  nerek w jednej sesji obrazowej. Jednakże brakuje metod, które umożliwiłyby oszacowanie GFR z~obrazów DCE-MRI bez ingerencji operatora na pewnym z etapów przetwarzania~obrazów.

Działania przedstawione w przedłożonej pracy są częścią projektu mającego na celu opracowanie komputerowej metody pomiaru GFR bezpośrednio z obrazów DCE-MRI, będącej na tyle dokładną, szybką i efektywną, aby znaleźć zastosowanie kliniczne. Celem pracy było stworzenie biblioteki w języku Python, umożliwiającej komputerową analizę ilościową obrazów DCE-MRI oraz porównanie różnych modeli farmakokinetycznych (PK) w~zastosowaniu do oceny wydolności nerek.

W pierwszej kolejności zaznaczono obszary obu nerek oraz aorty na obrazie DCE-MRI po korekcji ruchu, a następnie w celu usunięcia obszaru miedniczki nerkowej przeprowadzono segmentację na podstawie przebiegów czasowych intensywności sygnału poszczególnych wokseli posługując się analizą głównych składowych oraz algorytmem k-średnich. Średnie przebiegi czasowe stężenia środka cieniującego w obszarze funkcjonalnym każdej z nerek dopasowano do 4 modeli PK: \textit{Tofts and Kermode} (TK), \textit{extended Tofts and Kermode} (ETK), \textit{Patlak-Rutland} (PR) and \textit{two-compartment exchange model} (2CXM). Na podstawie otrzymanych parametrów wyliczono GFR dla każdej z nerek oraz całkowity GFR.

Opracowaną metodę przetestowano na sekwencjach DCE-MRI dziesięciu zdro\-wych osób, a otrzymane wartości całkowitego GFR porównano z wartościami otrzymanymi metodami analizy klirensu iohexolu oraz stężenia kreatyniny. 
Otrzymane wyniki pokazały, że 2CXM jest najdokładniejszym oraz najbardziej precyzyjnym z modeli, w~odniesieniu do analizy klirensu iohexolu, a rezultaty otrzymane z jego zastosowaniem są zbliżone do analizy stężenia kreatyniny.

Podsumowując stwierdzono, że 2CXM może być użyty w ostatnim kroku docelowej metody estymującej GFR z obrazów DCE-MRI, a stworzona biblioteka umożliwia jego zaimplementowanie. 
 

	\end{small}
	
	\vfill
		\normalsize \noindent \textbf{Słowa kluczowe:} DCE-MRI, nerka, współczynnik przesączania kłębuszkowego, GFR, modelowanie farmakokinetyczne, analiza ilościowa, segmentacja nerek.  
		
			\end{spacing}
	%\vspace{-1cm}
		\newpage
		\restoregeometry
\thispagestyle{empty}
\mbox{}

	\newpage	
