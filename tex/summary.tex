\chapter*{Summary}

\addcontentsline{toc}{chapter}{Summary}

The main purposes of this thesis were, firstly, to design a library in Python programming language for pharmacokinetic modelling from DCE-MRI images, which is to be incorporated into the fast method of GFR estimation, and secondly, to compare the performance of the few PK models for application of assessing renal function and to examine whether any of them can be considered accurate and precise enough to be used in the target method.

As a first step, DCE-MRI sequences were registered in a time domain and both kidneys as well as aorta were manually labelled. Next, the renal pelvis was removed from the kidneys' labels on the basis of voxels' intensity time courses. For this purpose the PCA together with k-means clustering was applied. Subsequently, the average time-intensity curves were obtained for both kidneys and aorta. The signal intensity was converted into concentration by finding the temporal point, at which positive $S'(t)$ reaches the maximum value. So obtained concentration time courses were then fitted to four pharmacokinetic models: Tofts Kermode, Extended Tofts Kermode, Rutland Patlak and two-compartment exchange models. Finally, on the basis of the obtained parameters the SKGFR as well as total GFR were calculated and compared with two chemical methods. 
  
The developed algorithm was tested on DCE-MRI sequence of ten different participants. Dice similarity coefficient of the pelvis segmentation was equal to DSC~=~0.86\,$\pm$\,0.06. For every of the sequence the pelvis was removed correctly without the need of manual correction. Regarding the performance of different PK models, the obtained results show that 2CXM is the most accurate and precise one and gives results very . What is more the obtained $P30 = 100\%$ for this model proves that its estimation is sufficient to be used in clinical use.   

All in all, the thesis was a big success. An important contribution is creating the module in Python, which will be used in further research on fast method of GFR estimation.  
On the basis of the drawn conclusions of practical tests the 2CXM model can be applied as a final step of the target method.
